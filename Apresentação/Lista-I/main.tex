%!TEX encoding = UTF-8 Unicode
%!TEX TS-program = xelatex
%% This template licensed under CC-BY-NC-SA by Koenraad De Smedt
\documentclass[a4paper,12pt]{article}
%\usepackage[margin=24mm]{geometry}

\usepackage{setspace}

\usepackage{fancyhdr}
\pagestyle{fancy}
\lhead{Centro Federal de Educação Tecnológica de Minas Gerais}

\cfoot{\thepage} %número da página
\rfoot{Timóteo 2015}

\usepackage[left=2.6cm,right=2.5cm,top=2.5cm,bottom=2.5cm]{geometry}

%\usepackage[utf8]{inputenc}
\usepackage[portuguese]{babel} 
\usepackage[T1]{fontenc}

\usepackage{fontspec,xltxtra,polyglossia,titling,graphicx}
\usepackage{verbatim,gb4e,synttree,multicol} % choose or add what you need
\usepackage[colorlinks,urlcolor=blue,citecolor=blue,linkcolor=blue]{hyperref}
\setmainfont[Mapping=tex-text]{Times New Roman} % or another similar font

\setdefaultlanguage{english}
\setotherlanguages{norsk}
\usepackage{natbib}
\bibliographystyle{plain}




\begin{document}
	\LARGE
	Resenha Crítica do Filme O Congresso Futurista
	
	
	
	\large
	
	\subsection*{\textbf{CONGRESSO Futurista, O.} Direção: Ari Folman. Produção: Estados Unidos, 2013, 122 min.}
	
	O filme O Congresso Futurista foi baseado na obra O Congresso Futurológico de Stanislav Lem. O filme possue partes em \textit{live-action} e outras em animação. Ele conta a história de uma atriz que recebeu uma proposta para assinar um contrato no qual ela concordava em ter sua imagem digitalizada para um computador. Anos depois ela retorna ao estúdio para participar de um congresso e renovar a assinatura do contrato. E neste congresso ela se depara com a atual situação cinematográfica.
	
	Na cidade onde ela estava as pessoas consumiam substâncias que causam alucinações que as fazem ver um mundo fantástico e animado. A atriz então se espanta com tal situação e no momento de seu discurso ela critica a busca daquelas pessoas por esconder a situação do mundo real atrás de um mundo imaginário. 
	
	Posteriormente após inalar um gás tóxico a personagem perde o controle das suas alucinações e para sobreviver da intoxicação os médicos a congelam até que a medicina evolua e haja como curá-la.
	Durante este tempo muita coisa acontece no mundo e o uso dos alucinógenos se espalha. Quando ela é descongelada seu maior desejo é reencontrar se filho, é então que ela ganha um cápsula capaz de remover o efeito do alucinógeno. Ao acordar descobre que o mundo real está um caos, num cenário pós-apocalíptico e que quase todas as pessoas escolheram viver num mundo de alucinações à realidade e descobre ainda que seu filho decidiu encontrá-la no mundo imaginário então ela decide voltar a imaginação para reencontrá-lo.
	
	Neste filme podem ser identificadas algumas tecnologias como: a criação de filmes através de recursos gráficos e conservação de cadáveres congelando-os. 
	
	Com a evolução tecnológica temos benefícios e malefícios. Substituir pessoas nos \textit{sets} de filmagem por programas de computador, faz com que haja a desumanização do cinema, já que máquinas não possuem sentimentos, e os atores não tem mais controle sobre sua imagem transmitida pelas mídias. A imagem do ser humano se torna um objeto de uso comercial, sem terem direito de escolha de qual filme gostaria de atuar. A tecnologia foi criada para beneficiar o homem e não substituí-lo, desta maneira o computador não pode ser mais importante que o homem, nem ocupar seu lugar. A tecnologia não deve evoluir ao ponto de ser mais que seu criador ou isso pode causar um grande problema.
	
	O homem é senhor da tecnologia e da ciência até quando estas necessitam da interferência dele para existir. A partir do momento que a tecnologia for capaz de se recriar e transformar e fazer ciência entao ela terá ultrapassado o nível humano e tomará controle sobre o ser humano tornando este um servo da tecnologia. Mas mesmo antes da tecnologia ser mais inteligente que o homem, ele se faz servo da tecnologia quando deixa que a tecnologia faça o que apenas o homem é capaz de fazer que é pensar, raciocinar e tomar decisões.
	
	o diretor deste filme é Ari Folman nascido em 17 de dezembro de 1962 é um israelita diretor de cinema, roteirista e trilha sonora compositor.
	
	Entre 1991 e 1996 ele fez vários documentários para a televisão israelense, especialmente nos territórios ocupados. Em 1996, ele escreveu e dirigiu Santa Clara, longa baseado em um romance de escritor tcheco Pavel Kohout. O filme ganhou vários prêmios em Israel, incluindo Melhor Filme e Melhor Diretor, entre outros. Ele também ganhou o Prêmio do Público no Festival de Berlim.
	
	Em sua carreira produziu várias séries de sucesso e um segundo longa-metragem, Made in Israel (2000), uma história futurista sobre a busca pela última nazista. Sua primeira incursão no mundo da animação vem com a série documental Os materiais que o amor é feito(2004). Nos primeiros três minutos de cada capítulo animado, os cientistas colocaram suas teorias sobre a evolução do amor.
	
	Em 2008, outro filme de animação chamado Valsa com Bashir, ambientados na Guerra do Líbano, permitiu a Folman ganhar vários prémios internacionais, incluindo o Globo de Ouro de melhor filme em língua estrangeira. Suas principais produções são Valsa com Bashir(2008),  O Congresso Futurista(2013).
	
	Eu não recomendaria este filme porque ele é de difícil compreensão e o enredo não é claro, deixando em aberto muitas informações importantes.
	\\
	
	\begin{center}
		\centering{\textbf{Leandro de Oliveira Pinto \\   Graduando de Engenharia da Computação.}
		}
	\end{center}
	
	\section*{Referências Bibliográficas}
	
	\textbf{Ari Folman,} Disponível em:<http://biografias.estamosrodando.com/ari-folman/> acesso em: 14 de novembro de 2015. 
	
	
	
\end{document}