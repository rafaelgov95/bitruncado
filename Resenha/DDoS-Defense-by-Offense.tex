%!TEX encoding = UTF-8 Unicode
%!TEX TS-program = xelatex
%% This template licensed under CC-BY-NC-SA by Koenraad De Smedt
\documentclass[a4paper,12pt]{article}
%\usepackage[margin=24mm]{geometry}

\usepackage{setspace}

\usepackage{fancyhdr}
\pagestyle{fancy}
\lhead{Universidade Federal de Mato Grosso do Sul}

\cfoot{\thepage} %número da página
\rfoot{Timóteo 2015}

\usepackage[left=2.6cm,right=2.5cm,top=2.5cm,bottom=2.5cm]{geometry}

%\usepackage[utf8]{inputenc}
\usepackage[portuguese]{babel} 
\usepackage[T1]{fontenc}

\usepackage{fontspec,xltxtra,polyglossia,titling,graphicx}
\usepackage{verbatim,gb4e,synttree,multicol} % choose or add what you need
\usepackage[colorlinks,urlcolor=blue,citecolor=blue,linkcolor=blue]{hyperref}
\setmainfont[Mapping=tex-text]{Times New Roman} % or another similar font

\setdefaultlanguage{english}
\setotherlanguages{norsk}
\usepackage{natbib}
\bibliographystyle{plain}

 


\begin{document}
\LARGE
Resenha Crítica do Artigo DDoS Defense by Offense



\large

\subsection*{\textbf{ACM SIGCOMM } Computer Communication Review - Proceedings of the 2006 conference on Applications, technologies, architectures, and protocols for computer communications, Pages 303-314. }

O artigo DDoS Defense by Offense dos autores Michael Walfish, Mythili Vutukuru, Hari Balakrishnan, David Karger  e  Scott Shenker do MIT (Instituto de Tecnologia Massachusetts), apresentam a concepção, implementação, analise e avaliação experimental de speak-up, uma defesa contra ataques distribuídos de negação de serviço (DDoS).

	Nas duas primeiras páginas do artigo os autores propoem uma defesa para servidores contra nivel de aplicação DDoS, o speak-up onde os clientes (legitimos e não legitimos ) são incentivados a enviar mais tráfego para um servidor atacado, esse conceito foi analisado ao decorrer do artigo

	 Com o speak-up, um servidotr vitimado incentiva todos os clientes, a  enviar automaticamente maiores volumes de tráfego, supondo que os atacantes já estão usando a maior parte de sua banda de upload por isso não pode reagir ao estímulo. Bons clientes, no entanto, tem largura de banca de upload de reposição e vai reagir ao estimolo com volumes drasticamente mais elevados tráfego fim a fim.\\

    
    Co em que os atacantes paralisam um servidor enviando solicitações aparentemente legítimas que consomem o recursos computacioais (como ciclios CPU, disco, memória entre outros).

\begin{center}
\centering{\textbf{Rafael Gonçaves de Oliveira Viana \\ Ramon Santos \\   Graduandos de Sistemas de Informação.}
}
\end{center}

\section*{Referências Bibliográficas}

\textbf{ACM SIGCOMM, DDoS Defense by Offense} Disponível em:<http://biografias.estamosrodando.com/ari-folman/> acesso em: 14 de novembro de 2015. 



\end{document}